\input{../../entete/entete-papier.tex}


\begin{document}


\renewcommand{\thetheoreme}{\!} % pas de num�ros au theo
\renewcommand{\thedefinition}{\!}
\renewcommand{\theremarque}{\!}
\renewcommand{\theexemple}{\!}
\renewcommand{\theexemples}{\!}
\renewcommand{\thepropriete}{\!}
\renewcommand{\themethode}{\!}
\renewcommand{\theexercice}{\!}
\renewcommand{\therappel}{\!}
\renewcommand{\thenotation}{\!}
\renewcommand{\thevocabulaire}{\!}

\renewcommand{\thesubsection}{\alph{subsection})}% lettre pour les sous-sections

%%%%%%%%%%%%%%%%%%%%%%%%%%%%%%%%%%%%%%%%%%%%%%%%%%%%%%%%%%%%%%%%%%%%%%

\begin{center}
\Large{Projet Sudoku}
\end{center}


%%%%%%%%%%%%%%%%%%%%%%%%%%%%%%%%%%%%%%%%%%%%%%%%%%%%%%%%%%%%%%%%%%%%%%%%%%%%%%%%%%%%%%%%%%%%%
 \section{G�n�ration d'une grille de sudoku}
%%%%%%%%%%%%%%%%%%%%%%%%%%%%%%%%%%%%%%%%%%%%%%%%%%%%%%%%%%%%%%%%%%%%%%%%%%%%%%%%%%%%%%%%%%%%%


%%%%%%%%%%%%%%%%%%%%%%%%%%
\subsection{Id�e de m�lange � partir d'une grille connue}



On peut consid�rer que presque toutes les grilles de m�me difficult� de sudoku peuvent �tre obtenues en m�langeant une grille connue.
Pour cela il faut m�langer les nombres mais aussi les lignes (par rang de 3) , les colonnes puis les blocs lignes et blocs colonnes ( en d�pla�ant trois lignes � la fois).

Le but premier est donc de cr�er des fonctions de m�lange.

Notre probl�me est pour effectuer l'�change entre deux lignes ou deux colonnes, en effet l'indexation des tables en python donne une erreur quand on fait une copie sur place � l'aide d'une variable temporaire. Il faut donc utiliser \texttt{deepcopy} pour pallier le probl�me.



%%%%%%%%%%%%%%%%%%%%%%%%%%%%%%%%%%%%%%%%%%%%%%
\subsection{Codage de la grille et des possibilit�s}


Pour stocker toutes les possibilit�s d'une case, on prend un 9-uplet bool�en : 0 si le nombre n'est pas possible et 1 s'il l'est. L'avantage est que le nombre associ� � la place dans le 9-uplet peut changer sans modifier les tests de r�solution.

On va donc travailler sur des tableaux de tableaux de tableaux !

%%%%%%%%%%%%%%%%%%%%%%%%%%%%%%%%%%%%%%%%%%%%%%%

\subsection{Test de validit� et difficult� d'une grille}

On peut r�soudre facilement un sudoku par un algorithme glouton en testant toutes les possibilit�s. Le probl�me  est d'�valuer la difficult� des grilles produites. 

Pour nous faciliter la t�che on va travailler sur une grille connue et m�langer apr�s avoir enlever les nombres. Cela a l'avantage de produire une nombre important de grille de m�me difficult�. On va classer les techniques de r�solution par difficult� :

\begin{itemize} 
\item Facile :
\begin{itemize}
\item Unique candidat par intersection ligne colonne et bloc.
\item Unique candidat par �limination dans les autres cases
\end{itemize}
\item Moyen :
\begin{itemize}
\item Obligation d'un nombre dans un bloc et sur une ligne (sur au moins deux cases) : �limination de ce nombre sur le reste de la ligne, colonne, bloc.
\item Deux candidats dans exactement deux cases possibles (dans un bloc, une ligne ou une colonne) : �limination de ces candidats ailleurs.
\end{itemize}
\item Difficile :
\begin{itemize}
\item Trois candidats dans exactement trois cases possibles : �limination de ce nombre sur le reste de la ligne, colonne, bloc.
\item Test de tous les candidats amenant � l'exclusion d'un nombre dans une case. 
\item Test al�atoire (algo glouton)
\end{itemize}

\end{itemize}

On va compter le nombre de fois que chaque technique est utilis�e, on utilise une technique plus dure si les simples ne fonctionnent pas. De plus on va pond�rer par le nombre de cases restantes � trouver.









\end{document}